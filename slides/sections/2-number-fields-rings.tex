\section{Number Fields and Number Rings}

\note{But all fields aren't created equal.}

\begin{frame}
	\frametitle{Number Fields}

	\begin{definition}[Number Field]
		A \emph{number field} $K \subset \C$ is a finite extension of $\Q$.
	\end{definition}

	\pause

	\begin{theorem}
		Any number field can be written in the form
		\begin{equation*}
			K
			= \Q[\alpha]
			= \spn_\Q \{ 1, \alpha, \alpha^2, \ldots, \alpha^{n-1} \}
		\end{equation*}
		where $n$ is the degree of the minimal polynomial of $\alpha$.
	\end{theorem}

	\pause

	\textbf{Examples}

	\begin{itemize}[<+->]
		\item $\Q[i \sqrt 5] = \{ a + i b \sqrt 5 : a, b \in \Q \}$
		\item $\Q[\omega]$, $\omega = e^{2 \pi i / p}$ (cyclotomic fields)
		\item $\Q[\sqrt m]$, $m$ squarefree (quadratic fields)
	\end{itemize}

	\pause

	\textbf{Non-examples}

	\begin{itemize}
		\item $\Q[\pi]$ because $\pi$ is transcendental
	\end{itemize}
\end{frame}

\begin{frame}
	\frametitle{Number Rings}

	\pause

	\note{We need an analogue to the integers in the complex plane}

	\begin{definition}[Algebraic Integer]
		An \emph{algebraic integer} is a complex number $\alpha$ that is the root of some monic polynomial $f \in \Z[x]$.

		\pause \medskip

		We use $\mathbb A$ to denote the set of all algebraic integers.
	\end{definition}

	\pause \medskip

	\begin{itemize}[<+->]
		\item Any integer $k$ is an algebraic integer because of $f(x) = x - k$
		\item $i \sqrt 5 \in \mathbb A$ because of $f(x) = x^2 + 5$
		\item $2 + \sqrt[3]{17} \in \mathbb A$ because of $f(x) = x^3 - 6 x^2 + 12 x - 25$
		\item $\mathbb A$ is a subring of $\C$
		      \note{show that $\alpha - \beta, \alpha \beta \in \mathbb A$}
	\end{itemize}

\end{frame}

\begin{frame}
	\frametitle{Number Rings}

	\begin{definition}[Number Ring]
		The \emph{number ring} of a number field $K = \Q(\alpha)$ is the set
		\begin{equation*}
			\mathcal O_K = \mathbb A \cap K.
		\end{equation*}
	\end{definition}

	\bigskip \pause

	\begin{itemize}[<+->]
		\item $\mathcal O_{\Q} = \mathbb A \cap \Q = \Z$ % TODO: why is this true?
		      \note{This was not immediately obvious to me. true because leading coef of minimal poly isn't 1}
		\item $\mathcal O_{\Q[i]} = \Z[i]$
		\item $\mathcal O_{\Q[\zeta]} = \Z[\zeta]$ for primitive roots $\zeta$
		      \note{But number rings aren't always this nice, or even one-dimensional.}
		\item $\mathcal O_{\Q[\sqrt 5]} = \Z [1, \frac12 + \frac12 \sqrt 5]$
	\end{itemize}

	\note{It turns out that these things have some nice properties. So nice in fact that mathemeticians have done what they love most and come up with another definition.}
\end{frame}

\begin{frame}
	\frametitle{Dedekind Domains}

	\begin{theorem}
		Number rings are Dedekind domains.
	\end{theorem}

	\begin{definition}[Dedekind domain]
		A \emph{Dedekind domain} is an integrally closed domain $R$ such that
		\begin{enumerate}
			\item every ideal is finitely generated and
			      \note{this is the same as Noetherian}
			\item every nonzero prime ideal is maximal.
			      \note{this is the same as krull dimension 1}
		\end{enumerate}
	\end{definition}

	\note{This is nice, but the real reason that we care about this is the following: Even though }

\end{frame}

\begin{frame}
	\frametitle{Dedekind Domains}
	\framesubtitle{Factoring ideals}

	\begin{theorem}
		Every ideal of a Dedekind domain $R$ uniquely factors into prime ideals.
	\end{theorem}

	\bigskip

	\begin{itemize}
		\item<2-> Allows replacing unique factorization of elements with unique factorization of ideals.
		\item<3-> In $R = \Z[i \sqrt 5]$,
			\begin{align*}
				6   & = {\color<5>{red}{2}}
				\cdot \color<6>{red}{3}                                                \\
				    & = {\color<7>{red}{(1 + i \sqrt{5})}}
				{\color<8>{red}{(1 - i \sqrt{5})}}                                     \\[1em]
				\onslide<4->{
				(6) & = {\color<5,7,8>{red}{(2, 1 + i \sqrt5)}} {}^{\color<5>{red}{2}}
				\, {\color<6,7>{red}{(3, 1 + i \sqrt5)}}
				\, {\color<6,8>{red}{(3, 1 - i \sqrt5)}}
				}
			\end{align*}
	\end{itemize}

	\note{In this way, the unique factorization of 6, the ideal, totally describes the failure of unique factorization of 6, the number}
\end{frame}

\begin{frame}
	\frametitle{Dedekind Domains}
	\framesubtitle{In the class hierarchy}

	\begin{theorem}
		A Dedekind domain has unique factorization iff it's a PID.
	\end{theorem}

	\bigskip \pause

	\note{Enables an alternate path from ICD to PID}

	% TODO: highlight left/right path as bullets come up on the right

	\begin{columns}
		\begin{column}{.5\textwidth}
			\begin{center}
				\begin{tikzpicture}
					\graph[
					layered layout, sibling distance=6em, level distance=4em,
					edges = {
							shorten >=2pt,
							shorten <=2pt
						},
					]{
					"Integrally Closed Domains"
					-> {UFDs, "Dedekind Domains"}
					-> PIDs;
					};
				\end{tikzpicture}
			\end{center}
		\end{column}

		\pause

		\note{Say you're an ICD and you want to be PID. You can either be a UFD or a DD, but not both (otherwise you become a PID)}

		\note{You have two options as an integrally closed domain with upward aspirations}

		\begin{column}{.5\textwidth}
			\begin{itemize}[<+->]
				% TODO: show that unique factorization of ideals fails in UFD?

				% https://math.stackexchange.com/questions/1669093/example-of-a-ufd-that-is-not-dedekind
				\item \textbf{DD, not UFD} \\
				      $\Z[i \sqrt5]$

				\item \textbf{UFD, not DD} \\
				      $\R[x_1, x_2, \ldots]$, $\Q[x, y]$
				      \bigskip
			\end{itemize}
		\end{column}
	\end{columns}

	\note{Okay, so we've seen that Dedekind domains have something to do with unique factorization, and this next section will make that connection a little bit more concrete.}
\end{frame}
