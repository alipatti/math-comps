\section{Chapter 1}

Algebraic number theory is the study of number fields: finite extensions of the rationals $\Q(a_1, \ldots, a_n)$.

\begin{example}
	The extension $\Q[i]$
\end{example}

\begin{theorem}[Fermat's Last Theorem]
	For $n > 2$, the equation $x^n + y^n = z^n$ has no solutions.
\end{theorem}

\begin{definition}[Class number]
	Let $p$ be a prime and take $\omega$ to be the $p$th root of unity $e^{2 \pi i / p}$.
	Then, the \em{class number} of the ring $\Z[\omega]$ (or the \em{class number} of $p$) is the number of equivalence classes under the following relation on the ideals of $\Z[\omega]$:
	\begin{center}
		$A \sim B$ iff $\alpha A = \beta B$ for some $\alpha, \beta \in \Z[\omega]$.
	\end{center}
	Let $h : \P \to \N$ be the function that gives the class number for a prime $p$.
\end{definition}

The relation $\sim$ above is an equivalence relation.

\begin{example}
	% TODO: give example
\end{example}

\begin{definition}[Regular primes]
	A prime $p$ is \em{regular} if $p \nmid h$
\end{definition}

\begin{definition}[Ideal class group]

\end{definition}

