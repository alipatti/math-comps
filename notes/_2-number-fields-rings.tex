\section{Number Fields and Number Rings}

\begin{definition}[Number field]
	A \em{number field} is a field $K \subset \C$ with $\deg_\Q K$ finite.
\end{definition}

\begin{theorem}
	Every number field has the form $\Q[\alpha]$ for some algebraic number $\alpha \in \C$.Furthermore, if $d$ is the degree of the minimal polynomial of $\alpha$, then the set $\{1, \alpha, \alpha^2, \ldots, \alpha^{d-1} \}$ is a basis for $\Q[\alpha]$.
\end{theorem}

\begin{proof}
	Given in Appendix B of Marcus.
\end{proof}

\begin{definition}[Cyclomatic field]
	Let $\omega_p = e^{2 \pi i / m}$ for some $m \in \N$. Then, the field $\Q[\omega_p]$ is the \em{$m$th cyclomatic field}.
\end{definition}

\begin{theorem}
	The degree of the $m$the cyclomatic field is $\phi(m)$.
\end{theorem}

\begin{definition}[Quadratic field]
	A \em{quadratic field} is a field of the form $\Q[\sqrt{m}]$ for any non-square $m \in \Z$. If $m < 0$, we call $Q[\sqrt{m}]$ an imaginary quadratic field. If $m > 0$, we call $Q[\sqrt{m}]$ a real quadratic field.
\end{definition}

\begin{theorem}
	All quadratic fields have degree $2$ over $\Q$ with basis $\{ 1, \sqrt{m} \}$.
\end{theorem}

\begin{theorem}
	Quadratic fields for squarefree $m$ are all distinct.
\end{theorem}

\begin{example}
	$\Q[\sqrt{-3}] = Q[\omega_6]$
\end{example}

\begin{definition}[Algebraic integer]
	A number $\alpha \in \C$ is an \em{algebraic integer} if it's the root of a monic polynomial $f \in \Z[x]$.
	We denote the sum of all algebraic integers as $\mathcal A$.
\end{definition}

\begin{theorem}
	Let $\alpha$ be an algebraic integer with a monic vanishing polynomial $f \in \Z[x]$ of minimal degree. Then $f$ is irredicible over $\Q$.
\end{theorem}

\begin{theorem}
	The followign are equivalent for $a \in \C$
	\begin{enumerate}
		\item $\alpha$ is an algebraic integer,
		\item The (addative) group $\Z[\alpha]$ is finitely generated,
		\item $\alpha$ is a member of some subring $R$ of $\C$ where $(R, +)$ is finitely generated,
		\item $\alpha A \subset A$ for some finitely-generated additive subgroup $A \subset \C$.
	\end{enumerate}
\end{theorem}

\begin{corrolary}
	If $\alpha, \beta \in \A$, then $\alpha + \beta, \alpha \beta \in \A$.
\end{corrolary}

\begin{definition}[Number ring]
	The \em{number ring} of a number field $K$ is the ring $R = \mathbb A \cap K$.
\end{definition}

\begin{example}
	The corresponding number ring for the cyclomatic field $\Q[\omega]$ is $\A \cup \Q[\omega] = \Z[\omega]$.
\end{example}

\begin{definition}[Embeddings]
	An \textit{embedding} of a field $K$ in a field $F$ is a ring homomorphism $\sigma : K \to F$.
\end{definition}

\begin{theorem}
	Any embedding is injective.
\end{theorem}

\begin{definition}[Trace and norm]
	Let $K$ be a number field with embeddings $\sigma_1, \ldots, \sigma_n : K \to \C$ with $n = [K : \C]$. Then, the \textit{trace} of some element $\alpha \in K$ is
	\begin{equation}
		T(\alpha) = \sigma_1(\alpha) + \cdots \sigma_n(\alpha)
	\end{equation}
	and the \textit{norm} is
	\begin{equation}
		N(\alpha) = \sigma_1(\alpha) \cdots \sigma_n(\alpha).
	\end{equation}
\end{definition}

\begin{definition}
	\begin{equation}
		t(n) =
	\end{equation}
\end{definition}

\begin{theorem}
	Let $K$ be a number field with an element $\alpha \in K$ Let $d$ be the degree of $\alpha$ and $n = [K : \C]$. Then,
	\begin{equation}
		T(\alpha) = \frac n d t (\alpha)
	\end{equation}
	and
	\begin{equation}
		N(\alpha) = (n(\alpha))^{n / d}.
	\end{equation}
\end{theorem}

\begin{corrolary}
	$T(\alpha)$ and $N(\alpha)$ are rational.
\end{corrolary}

\begin{corrolary}
	If $\alpha$ is an algebraic integer, then $T(\alpha)$ and $N(\alpha)$ are integers.
\end{corrolary}

\begin{definition}[Relative trace and norm]
	Let $K \subset L$ be two number fields and let $\sigma_1, \ldots, \sigma_n$ be the $n = [K : L]$ embeddings of $L$ in $\C$ that fix $K$ pointwise. Then, for $\alpha \in L$, the \textit{relative trace} is
	\begin{equation}
		T^L_K(\alpha) = \sigma_1(\alpha) + \cdots + \sigma_n(\alpha)
	\end{equation}
	and the \textit{relative norm} is
	\begin{equation}
		N^L_K(\alpha) = \sigma_1(\alpha) \cdots \sigma_n(\alpha).
	\end{equation}
\end{definition}

\begin{theorem}
	Let $\alpha \in L$ and take $d \ceq \deg_K(\alpha)$. Then,
	% TODO: finish
\end{theorem}

\begin{corrolary}
	The relative trace and norm of $\alpha$ are in $K$. If $\alpha \in \A \cap L$, then they are in $\alpha \cap K$.
\end{corrolary}

\begin{theorem}[Transitivity of the trace and norm]
	Let $K \subset L \subset M$ be number fields. Then, for all $\alpha \in M$,
	\begin{align}
		T_K^L(T_L^M(\alpha)) & = T_K^M(\alpha), \\
		N_K^L(N_L^M(\alpha)) & = N_K^M(\alpha).
	\end{align}
\end{theorem}

\begin{proof}
	% TODO: tex proof
\end{proof}

\begin{definition}[Discriminant]
	Let $K$ be a number field of with degree $n$ over $\Q$. Let $\sigma_1, \ldots, \sigma_n$ be the embeddings of $K$ in $\C$. Then, the discriminant of $\alpha_1, \ldots, \alpha_n \in K$ is
	\begin{equation}
		\begin{aligned}
			\disc(\alpha_1, \ldots, \alpha_n)
			 & = \det(\sigma_i(\alpha_j))^2
			 & = \det \begin{pmatrix}
				          \sigma_1(\alpha_1) & \cdots & \sigma_1(\alpha_n) \\
				          \vdots             & \ddots &                    \\
				          \sigma_n(\alpha_1) &        & \sigma_n(\alpha_n)
			          \end{pmatrix}^2.
		\end{aligned}
	\end{equation}
\end{definition}

\begin{theorem}[Discriminant with respect to the trace]
	\begin{equation}
		\disc(\alpha_1, \ldots, \alpha_n) = \det \left(T(a_i a_j)\right)_{ij}
	\end{equation}
\end{theorem}

\begin{theorem}[Discriminant of linearly dependent numbers]
	$\disc(\alpha_1, \ldots, \alpha_n) = 0$ iff $\alpha_1, \ldots, \alpha_n$ are linearly dependent.
\end{theorem}

