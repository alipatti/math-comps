% TODO: Use example of finding all primative pythagorean triples to introduce algebraic number theory

\begin{figure}
	\centering
	% TODO: show how some gaussian integer factors into primes
	\begin{tikzpicture}[scale=.7]
		\draw[->] (-4,0) -- (4,0) node[right] {$\Re$};
		\draw[->] (0,-4) -- (0,4) node[above] {$\Im$};

		\foreach \a in {-3,-2,-1,0,1,2,3}
			{
				\foreach \b in {-3,-2,-1,0,1,2,3}
					{
						\fill[gray] (\a, \b) circle (1.5pt);
					}
			}
	\end{tikzpicture}

	\caption{The lattice of Gaussian integers $\Z[i]$ in the complex plane.}
\end{figure}

% TODO: why do a and b have to be odd?

\begin{table}
	\centering
	\begin{tabular}{c|c|r}
		$a$ & $b$ &                      \\
		\hline
		$2$ & $1$ & $3^2 + 4^2 = 5^2$    \\
		$3$ & $2$ & $5^2 + 12^2 = 13^2$  \\
		% $3$ & $1$ & $8^2 + 6^2 = 10^2$   \\
		$4$ & $3$ & $7^2 + 24^2 = 25^2$  \\
		$4$ & $2$ & $12^2 + 16^2 = 20^2$ \\
		$4$ & $1$ & $15^2 + 8^2 = 17^2$
	\end{tabular}
	% TODO: add table of all pythagorean triples
	\caption{The primative Pythagorean triples.}
\end{table}

% TODO: note what was cool about this derevation

We started with a problem strictly about the integers, and

% TODO: use Marcus's definition of algebraic number theory

% TODO: introduce this quote

This paper will build towards proving that the ideal class group is finite, starting from Carleton's introductory algebra class. However, it will at times move very, very quickly and leave gaps in the exposition for the sake of brevity.
To quote Richard Feynman:

\begin{quote}
	I am going to give what I will call an elementary demonstration. But elementary does not mean easy to understand. Elementary means that very little is required to know ahead of time in order to understand it, except to have an infinite amount of intelligence.
	% TODO: cite
	% https://www.google.com/books/edition/_/o2VLdx2td0cC
	% p. 148
\end{quote}

If this paper piques your interest, I encourage you to go read \textcite{marcus}.
