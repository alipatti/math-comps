\section{The Ideal Class Group}

We noticed in the previous section that the unique prime factorization of ideals revealed information about the \emph{failure} of unique factorization of ring elements. This section gives a little more formality to that idea. We begin with the following construction:

\begin{definition}[Ideal Class Group]
    \label{def:ideal-class-group}
    Let $K = \Q[\alpha]$ be a number field. The \textit{class group} of $K$ is the set of ideals of $\mathcal O_K$, modulo the equivalence relation
    \begin{equation}
        I \sim J \text{ iff } \alpha I = \beta J \text{ for some nonzero } \alpha, \beta \in R.
    \end{equation}
\end{definition}

This happens to be the smallest equivalence relation that ``kills'' the principal ideals, i.e., collapses them into a single equivalence class.
This follows from the fact that one can analogously define the ideal class group as the quotient of fractional ideals by principal ideals.

Returning to our old friend $\Z[i \sqrt 5]$, one can verify that $(2) \sim (3)$ by the fact that $3 (2) = (6) = 2 (3)$.
In fact, for any number field $K$, any two principal ideals $(a), (b) \subseteq \mathcal O_K$ are equivalent by the fact that $a (b) = (ab) = b (a)$.
This equivalence class of principal ideals plays the role of the identity element in the class group which is---as the name suggests---a group. The group operation given by ideal multiplication of some representative element from each ideal class. We leave it to the reader to verify that this operation is well-defined. Inverses are guaranteed to exist by \autoref{thm:ideal-inverses-exist} which guarantees that for any ideal $I$, there exists some $J$ such that $IJ$ is principal.

What about the non-principal ideals of $\Z[i \sqrt 5]$? We saw that the ideal $(6)$ factored into four prime ideals; under $\sim$, they're all equivalent:
\begin{equation}
    \begin{aligned}
        (6, 3 + 3 i \sqrt 5)
         & = 3 \, (2, 1 + i \sqrt 5)                \\
         & = 3 \, (2, 1 - i \sqrt 5)                \\
         & = (1 - i \sqrt 5) \, (3, 1 + i \sqrt 5)  \\
         & = (1 + i \sqrt 5) \, (3, 1 - i \sqrt 5).
    \end{aligned}
\end{equation}
(In the last two lines, the left-hand terms of the multiplication are elements of $\Z[i \sqrt 5]$, not ideals.)

It turns out that these are the only two equivalence classes, so we say that the class group of $\Q[i \sqrt 5]$ is (isomorphic to) the cyclic group $\mathcal C_2$.
The principal ideals play the role of the identity element $1$; the non-principal ideals play the role of $-1$.
We saw evidence of this earlier in \autoref{eqn:recovering-factors-of-6} when we were able to multiply the prime factors of $(6)$ to recover the irreducible factors of $6$.
We didn't know it at the time, but what we really were observing is that $(-1)^2 = 1$.

\subsection{The Class Number}

\begin{definition}[Class Number]
    The class number of a number field $K = \Q(\alpha)$, denoted $h(K)$, is the size of its ideal class group.
\end{definition}

Before we do anything too complicated, note that this new definition allows us to rephrase \autoref{thm:dedekind-ufd-implies-pid}:

\begin{theorem}
    The number ring $\mathcal O_K$ has unique factorization if and only if $h(K) = 1$.
\end{theorem}

\begin{proof}
    The class number $h(K) = 1$ iff all ideals in $\mathcal O_K$ are principal, i.e., iff $\mathcal O_K$ is a PID. Number rings are Dedekind domains, so the result follows from \autoref{thm:dedekind-ufd-implies-pid}.
\end{proof}

We now proceed to the main goal of our paper. With so much machinery now built up, stating the theorem is quite simple:

\begin{theorem}
    \label{thm:clas-number-finite}
    Class numbers are always finite.
\end{theorem}

\begin{proof}
    We prove \autoref{thm:clas-number-finite} through a sequence of lemmas.

    \begin{lemma}
        Let $K$ be a number field and $\mathcal O_K = \A \cap K$ its field of integers. Then, there exists some $\lambda > 0$ such that every non-trivial ideal of $R$ contains a nonzero $\alpha$ such that
        \begin{equation}
            \label{eqn:ideal-norm-bound}
            \left| \Norm^K(\alpha) \right| \leq \lambda || I ||.
        \end{equation}
    \end{lemma}

    Take $\alpha_1, \ldots, \alpha_n$ to be an integral basis for $\mathcal O_K = \spn_\Z\{\alpha_1, \ldots, \alpha_n\}$ and let $\sigma_1, \ldots, \sigma_n$ be the embeddings of $K$ in $\C$. We claim that \autoref{eqn:ideal-norm-bound} holds for
    \begin{equation}
        \lambda = \prod_{i \leq n} \sum_{j \leq n} | \sigma_i(\alpha_j)|.
    \end{equation}
    To find $\alpha$, fix some ideal $I \subseteq R$, take $m$ to be the largest integer such that $m^n \leq ||I|| < (m+1)^n$ and consider the following set:
    \begin{equation}
        M = \left\{ \sum_{j \leq n} m_j \alpha_j : m_j = 0, 1, \ldots, m \right\}.
    \end{equation}
    To generate an element of $M$, one must make $n$ choices for $m_j$, each of which has $m+1$ options (the integers $0, \ldots, m$. So $|M| = (m+1)^n > ||I||$, and the pigeonhole principle guarantees that two elements of $M$ must be congruent modulo $I$---call them $a$ and $b$. We take
    \begin{equation}
        \label{eqn:class-group-proof-alpha}
        \alpha = b - a = \sum_{j \leq n} m_j \alpha_j
    \end{equation}
    for some integer $m_j$ with $m_j \leq m$. It follows that
    \begin{equation}
        \begin{aligned}
            \left| \Norm^K(\alpha) \right|
             & = \prod_{i \leq n} |\sigma_i(\alpha)|                           \\
             & \leq \prod_{i \leq n} \sum_{j \leq n} m_j | \sigma_i(\alpha_j)| \\
             & \leq \lambda m^n                                                \\
             & \leq \lambda || I ||.
        \end{aligned}
    \end{equation}

    \medskip

    \begin{lemma}
        Every equivalence class of $\mathcal O_R$ contains an ideal $J$ with $||J|| \leq \lambda$.
    \end{lemma}

    Let $C$ be an equivalence class of $O_R$ with respect to $\sim$ (from \autoref{def:ideal-class-group}). Take some $I \in \C \inv$ and define $\alpha$ as in \autoref{eqn:class-group-proof-alpha}. By construction, $\alpha \in I$, so $(\alpha) \subseteq I$ and therefore there must exist some ideal $J$ such that $IJ = (\alpha)$ by \autoref{thm:ideal-divisibility}. Because $IJ$ is principal and $I \in C \inv$, it must be that $J \in C$. Finally,
    \begin{equation}
        \lambda ||I||
        \geq |N^K(\alpha)|
        = ||(\alpha)||
        = ||I|| \, ||J||
    \end{equation}
    where the last two equalities follow from \cite[Theorem 22]{marcus}. So $||J|| \leq \lambda$.

    \medskip

    \begin{lemma}
        There are only finitely many ideals satisfying $||J|| \leq \lambda$.
    \end{lemma}

    Let $P_i$ be the prime divisors of $J$ with corresponding multiplicities $b_i$.
    Then,
    \begin{equation}
        \lambda
        \geq ||J||
        = ||\prod P_i^{n_i}||
        = \prod ||P_i||^{n_i}.
    \end{equation}
    There are only finitely many primes $P_i$ and multiplicities $b_i$ satisfying this inequality, so there are only finitely many possible $J$.
    % FIX: why is this true?

    \medskip

    Each ideal class must contain some ideal satisfying $||J|| \leq \lambda$, of which there are only finitely many. \autoref{thm:clas-number-finite} follows.
\end{proof}

We won't give any rigorous meaning to this claim, but it seems that the class number of a number field $K$ captures ``how far away'' a group is from achieving unique factorization.
In the previous section, we saw that the class number of $\Q[i \sqrt5]$ was $2$, so in some sense it couldn't be any closer to being a UFD.

There's an encouraging undertone here: If every number ring yearns for unique factorization, it's never infinitely far from its dreams.

% % FIX: is this worth discussing?
%
% \subsection{The Unit Theorem}
%
% \begin{theorem}[Unit theorem]
% \end{theorem}
%
% Proving this theorem is outside the scope of this paper---the proof alone in \textcite[100-103]{marcus} spills onto four pages and relies on tens of pages material that we haven't developed here.
%
% Even without proof, this theorem provides some explanation for why the class number of real quadratic fields is much harder to describe than that of imaginary quadratic fields. \ldots
%
% % explain why this makes real quadratic fields more complicated than imaginary ones?

