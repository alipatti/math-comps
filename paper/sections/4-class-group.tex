\section{The Ideal Class Group}

\begin{definition}[Ideal Class Group]
	Let $K = \Q[\alpha]$ be a number field. The \textit{class group} of $K$ is the set of ideals of $\mathcal O_K$, modulo the equivilence relation
	\begin{equation*}
		I \sim J \text{ iff } \alpha I = \beta J \text{ for some nonzero } \alpha, \beta \in R.
	\end{equation*}
\end{definition}

This is the smallest relation that "kills the principal ideals", i.e., collapses them into a single equivilence class.
You can also define the ideal class group as the quotent of fractional ideals by principal ideals.

\begin{theorem}
	The ideal class group is a group.
\end{theorem}

\begin{proof}
	\begin{lemma}[Ideal inverses exist]
		% TODO: copy proof from marcus
	\end{lemma}

	Have to show that there is an identity, inverses exists, and that ideal multiplication is well-defined.
\end{proof}

\subsection{The Class Number}

With so much machinery now built up, stating the goal of our paper is quite simple:

\begin{definition}[Class Number]
	The class number of a number field $K = \Q[\alpha]$ is the size of its ideal class group.
\end{definition}

\begin{theorem}
	Class numbers are always finite.
\end{theorem}

Never infinitely far from your dreams.

