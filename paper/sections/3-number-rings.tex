\section{Number Rings}

We have a generalization of the rationals to the complex plane, now we need an analogue to the integers.

\begin{definition}[Algebraic Integer]
	An \emph{algebraic integer} is a complex number $\alpha$ that is the root of some monic polynomial $f \in \Z[x]$.

	We use $\mathbb A$ to denote the set of all algebraic integers.
\end{definition}

We would hope that the familiar integers $\Z$ remain integers under this more general definition, and that is indeed the case.

$f(x) = x - k$

\begin{itemize}
	\item Any integer $k$ is an algebraic integer because of $f(x) = x - k$
	\item $i \sqrt 5 \in \mathbb A$ because of $f(x) = x^2 + 5$
	\item $2 + \sqrt[3]{17} \in \mathbb A$ because of $f(x) = x^3 - 6 x^2 + 12 x - 25$
	\item $\mathbb A$ is a subring of $\C$
	      \note{show that $\alpha - \beta, \alpha \beta \in \mathbb A$}
\end{itemize}

% TODO: give process to generate minimal $f$ for $\alpha + \beta$ and $\alpha \beta$?

\begin{definition}[Number Ring]
	The \emph{number ring} of a number field $K = \Q(\alpha)$ is the set of algebraic integers contained within $K$, denoted
	\begin{equation*}
		\mathcal O_K = \mathbb A \cap K.
	\end{equation*}
\end{definition}

We use $\mathcal O$ because it looks like a ring.

% TODO: give examples

Sometimes you can just move $\alpha$ from next to the $\Q$ to next to the $\Z$.

% - cyclotomic field
% - quadratic field
% - sqrt 5 (two dimensional integral basis, show why we can't just to $\Z[\sqrt 5]$
% - when is this the case?

\subsection{Dedekind Domains}

Number rings have some nice properties, so mathematicians have done what they love best and given those properties a definition.

\begin{definition}[Noetherian domain]

\end{definition}

\begin{definition}[Dedekind domain]
	A \emph{Dedekind domain} is a ring $R$ such that
	\begin{enumerate}
		\item $R$ is integrally closed,
		\item $R$ is Noetherian, and
		\item every nonzero prime ideal is maximal.
	\end{enumerate}
\end{definition}

\begin{theorem}[Number Rings are Dedekind Domains]

\end{theorem}

\begin{proof}

\end{proof}

\begin{figure}
	\centering
	% TODO: fix node positioning
	\begin{tikzpicture}
		\graph[
		layered layout, rotate=90,
		sibling distance=8em, level distance=5em,
		edges = { arrow },
		nodes = { box },
		]{
		"Integrally Closed \\ Domains"
		-> {UFDs, "Dedekind \\ Domains"}
		-> PIDs;

		"Noetherian \\ Domains" -> "Dedekind \\ Domains";

		};
	\end{tikzpicture}
	\caption{The two paths for an integrally closed domain with upward ambitions.}
\end{figure}

This definition is pretty unhelpful. The reason we care about Dedekind domains at all is the following.

\begin{theorem}
	\label{thm:dedekind-ideal-factorization}
	Ideals in Dedekind domains uniquely factor into prime ideals.
\end{theorem}

Although we won't prove it, the converse of \autoref{thm:dedekind-ideal-factorization} is also true, and is sometimes taken as an alternative definition of Dedekind domains.
% TODO: provide a citation for this

We use the one that we've chosen to use because its conditions are easier to show.

% TODO: why do we care?
% - replaces unique factorization of elements with unique factorization of rings

\begin{theorem}
	\label{thm:dedekind-ufd-implies-pid}
	If a ring is a Dedekind domain and a unique factorization domain, then it's a principal ideal domain.
\end{theorem}

Once you have prime factorization of both ring elements and ideals, you are guaranteed that all ideals are principal. Before we prove this theorem, we're going to work through an example.

% TODO: give example of factoring 6 in $\Z[i \sqrt 5]$

The unique factorization of the ideal $(6)$ completely captures the \emph{failure} of unique factorization of the ring element $6$. But this only happens because the prime ideals that $6$ factors into are generated by two elements.

\begin{proof}[Proof (of \autoref{thm:dedekind-ufd-implies-pid})
	% TODO: write proof
\end{proof}

% TODO: prove or mention that every ideal in a DD is generated by at most two elements?

% TODO: give examples of UFDs that aren't DDs

