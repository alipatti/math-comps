\section{Number Fields}

Enough beating around the bush, here's the definition

\begin{definition}[Number Field]
	A \emph{number field} $K \subset \C$ is a finite extension of $\Q$.
\end{definition}

\begin{definition}[Algebraic number]
	A complex number $\alpha \in \C$ is an \emph{algebraic number} if it is the root of some monic polynomial $f \in \Q[x]$.
\end{definition}

\begin{theorem}
	Any number field can be written in the form
	\begin{equation*}
		K
		= \Q[\alpha]
		= \spn_\Q \{ 1, \alpha, \alpha^2, \ldots, \alpha^{n-1} \}
	\end{equation*}
	where $n$ is the degree of the minimal polynomial of $\alpha$.
\end{theorem}

\begin{proof}
	% TODO: write proof
\end{proof}

Although we could use any algebraic number for $\alph$, for the purposes of this paper we'll focus on the two following families

\begin{definition}[Cyclotomic field]

\end{definition}

\begin{definition}[Quadratic field]

\end{definition}

It seems like positive $m$ should be the simpler case, but itt turns out we know very little about these when $m > 0$ (more on this later).

\subsection{Complex Embeddings}

\begin{figure}
	% TODO: make this diagram better, show something about embeddings?
	\centering
	\begin{tikzpicture}[scale=2]
		\draw[->] (-1.2,0) -- (1.2,0) node[right] {Re};
		\draw[->] (0,-1.2) -- (0,1.2) node[above] {Im};
		\foreach \i in {0,1,2,3,4}
			{
				\draw[red,->] (0,0) -- ({cos(360*\i/5)},{sin(360*\i/5)});
				\node[red,anchor=north west] at ({cos(360*\i/5)},{sin(360*\i/5)}) {$\omega^{\i}$};
			}
	\end{tikzpicture}
	\caption{The fifth roots of unity.}
\end{figure}

\begin{definition}[Trace and norm]

\end{definition}

\begin{definition}[Relative trace and norm]

\end{definition}

