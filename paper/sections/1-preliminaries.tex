\section{Preliminaries}

Algebraic number theory (unsurprisingly) requires a lot of algebra.

\subsection{Groups}

We assume that the reader is familiar with the definition of a group and provide it here only for completeness.

\begin{definition}[Group]
	A \textbf{group} is a set $G$ along with an operation $\cdot : G \times G \to G$ such that for all $a, b \in G$,
	\begin{enumerate}
		\item $a \cdot b \in G$ (closure),
		\item there exists an \emph{identity element} $e$ such that $a \cdot e = a$ (identity),
		\item there exists some element $a \inv$ such that $a \cdot a \inv = e$.
	\end{enumerate}
	If the $\cdot$ operation is commutative, we call $G$ an \textbf{Abelian group}.
\end{definition}

\subsection{Rings}

Much of algebraic number theory is concerned with generalizations of integers int he complex plane, and how these generalizations are and are not similar to the integers. This is done through the lens of rings.

\begin{figure}
	\centering
	\begin{tikzpicture}[
			node distance = 2em,
			start chain = going right,
			every node/.style = { on chain, join, box, font={\footnotesize}},
			every join/.style = {->,
					shorten > = 2pt,
					shorten < = 2pt},
		]
		% TODO: add links to pictures
		\node {Rings};
		\node {Integral \\ Domains};
		\node {Integrally \\ Closed \\ Domains};
		\node {Unique \\ Factorization \\ Domains};
		\node {Principal \\ Ideal \\ Domains};
		\node {Fields};
	\end{tikzpicture}
	\caption{The hierarchy of rings.} % TODO: better caption
\end{figure}

To do algebraic number theory, one unsupringly needs a bit of algebra.

Much of this (but not all!) is covered in Carleton's standard Algebra I course.

\begin{definition}[Commutative ring]
	% TODO: add this
\end{definition}

\begin{definition}[Ideal]
	An (additive) subgroup, $I$, such that $ra \in I$ for all $r \in R$, $a \in I$.
\end{definition}

\begin{definition}[Prime Ideal]
	An ideal, $P$, such that $ab \in P$ implies $a \in P$ or $b \in P$.
\end{definition}

% TODO: give examples of ideals
% TODO: give examples of prime ideals

% TODO: include ring hierarchy diagram

\begin{definition}
	% FIX: do we need this definition?
	A commutative ring is an \emph{integral domain} if $ab = 0$ implies $a = 0$ or $b = 0$. (No zero divisors.)
\end{definition}

\begin{definition}[Fraction field]

\end{definition}

% TODO: tell people to just go off vibes and trust the process

\begin{definition}[Integral elements]
	% TODO: write definition
\end{definition}

The classic example is that $\Frac \Z = \Q$.

\begin{definition}[Integrally Closed Domain]
	A ring $R$ is \textit{integrally closed} if for all $\alpha / \beta \in \Frac R$ that are integral over $R$, then $\beta \mid \alpha$, i.e., $\alpha / \beta \in R$.

	\note{We say that the integral closure of $\Frac R$ is $R$.}
\end{definition}

% TODO: unpack this definition

This will come up later in the paper.

Any monic polynomials with integer coefficients must have integer roots.

\begin{definition}[Unique factorization domain]
	A commutative ring $R$ is a \emph{unique factorization domain} if every element factors uniquely into irreducible elements.
\end{definition}

The integers are famously a UFD by the fundemental theorem of arithmatic.

\begin{equation}
	\begin{aligned}
		6 & = 2 \cdot 3                     \\
		  & = (1 + i \sqrt5) (1 - i \sqrt5)
	\end{aligned}
\end{equation}

The ring $\Z[i \sqrt 5]$ will make many appearances throughout the rest of this paper.

\begin{definition}[Principal Ideal Domain]
	A commutative ring $R$ is a \emph{principal ideal domain} if every ideal is generated by a single element.
\end{definition}

% TODO: give examples

The rings $\Z[x]$ and $\Q[x, y]$ are not because of the ideals $(2, x)$ and $(x, y)$ respectively.

\subsection{Fields}

\begin{definition}[Field]
	% TODO: give definition
\end{definition}

% use field extensions and the fact that Q isn't algebraically closed to pivot into discussion of number fields

\begin{equation}
	\begin{aligned}
		f(x) & = x^2 + 5                        \\
		     & = (x + i \sqrt 5)(x - i \sqrt 5)
	\end{aligned}
\end{equation}

In some sense, $f$ would be equally happy living in a much smaller field without these

\begin{definition}[Finite field extensions]
	The \emph{field extension} $K(\alpha_1, \ldots, \alpha_n)$ is the smallest field containing both $K$ and each of $\alpha_1, \ldots, \alpha_n$.
\end{definition}

This is a number field.
