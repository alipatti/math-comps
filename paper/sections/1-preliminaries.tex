\section{Preliminaries}
\label{sec:preliminaries}

Algebraic number theory (unsurprisingly) requires a lot of algebra---most texts require a graduate course or two, but I've attempted to condense the necessary background into the following few pages. See e.g. \textcite{dummit-foote} for more complete exposition.

\subsection{Groups}

We assume that the reader is familiar with the definition of a group and provide it here only for completeness.

\begin{definition}[Group]
    \label{def:group}
    A \textbf{group} is a set $G$ along with a binary operation \ $\cdot : G \times G \to G$ such that,
    \begin{enumerate}
        \item $a \cdot b \in G$ for all $a, b \in G$(closure),
        \item there exists an \emph{identity element} $e$ such that $a \cdot e = a$ for all $a \in G$ (identity),
        \item for every $a \in G$, there exists some element $a \inv$ such that $a \cdot a \inv = e$ (inverses).
    \end{enumerate}
    If the $\cdot$ operation is commutative, we call $G$ an \textbf{Abelian group}.
\end{definition}

\begin{definition}[Subgroup]
    We say that $H \subset G$ is a \emph{subgroup of $G$} (written $H \leq G$) if $H$ is a group with respect to the group operation of $G$.
\end{definition}

\subsection{Rings}

Much of algebraic number theory is concerned with generalizations of integers in the complex plane. This is done through the theory of rings.

\begin{definition}[Commutative ring]
    A \textbf{commutative ring} is a set $R$ with two commutative operations $+$ and $\cdot$ such that
    \begin{enumerate}
        \item $R$ is an Abelian group with respect to addition,
        \item $R$ is closed under multiplication, and
        \item multiplication distributes over addition.
    \end{enumerate}
\end{definition}

\begin{definition}[Ideal]
    An (additive) subgroup $I \leq R$ is an \textbf{ideal} if $ra \in I$ for all $r \in R$, $a \in I$.

    We define $A = (a_1, a_2, \ldots)$ to be the smallest ideal containing every $a_i$ and say that $A$ is \textbf{generated} by $\{a_1, a_2, \ldots \}$.
\end{definition}

We define multiplication of ideals as follows: If $A$ and $B$ are ideals, then their product is the ideal generated by the set
\begin{equation}
    \{ ab : a \in A, b \in B \}.
\end{equation}
For principal ideals, $(a) (b) = (ab)$, and the product of two non-principal, finitely generated ideals is
\begin{equation}
    (a_1, \ldots, a_n)(b_1, \ldots, b_n) = (a_i b_j : i \leq n, j \leq m).
\end{equation}
We leave it to the reader to show that if $A$ and $B$ are ideals, that $AB$ is also an ideal and that $AB \subset A \cap B$.

Much of number theory concerns prime numbers, so we extend the notion of primality to ideals in the following way:

\begin{definition}[Prime Ideal]
    An ideal $P$ is \textbf{prime} if $ab \in P$ implies $a \in P$ or $b \in P$.
\end{definition}

This is a generalization of Euclid's Lemma which states that $p$ is prime if and only if $p \mid ab$ implies $p \mid a$ or $p \mid b$. In the integers, prime ideals are those ideals generated by prime elements.

\begin{definition}
    A commutative ring is an \textbf{integral domain} if $ab = 0$ implies $a = 0$ or $b = 0$. (No zero divisors.)
\end{definition}

\begin{definition}[Fraction field]
    The \textbf{fraction field} of an integral domain $R$ is the smallest field containing $R$. It is equal to
    \begin{equation}
        \Frac R = \left\{ \frac a b : a, b \in R, b \neq 0 \right\} \big/ \sim
    \end{equation}
    where $\sim$ is the equivalence relation $\sfrac ab = \sfrac pq$ if $aq = bp$. (This is just your familiar cross-multiplication.)
\end{definition}

The classic example is that $\Frac \Z = \Q$; in other rings, the definition behaves very intuitively. One can almost always forget about the equivalence relation and just cancel like terms in the numerator and denominator. For example, the fraction field of the polynomial ring $\C[x]$ is the field of rational functions $\C(x)$.

\begin{definition}[Integral elements]
    \label{def:integral-element}
    Let $A$ be a ring a subring $B$. An element $a \in A$ is \textbf{integral over $B$} if $a$ is the root of some monic polynomial $f \in B[x]$. If $A \geq B = \Z$, we often drop the ``over $B$'' part and say that $f$ is \textbf{integral}.
\end{definition}

Any integer $a$ is trivially integral because of the polynomial $f_a(x) = x - a$, but there are more complicated examples too.
For example, $\frac12 + \frac12 \sqrt 5 \in \Z[\sqrt5]$ is integral by the polynomial $x^2 - x - 1$.

\begin{definition}[Integrally Closed Domain]
    \label{def:integrally-closed}
    Let $R$ be an integral domain. We say that $R$ is \textbf{integrally closed} if the fact that $a \in \Frac R$ is integral over $R$ implies that $a \in R$.

    Equivalently, $R$ is integrally closed if there are no elements of $\Frac R \setminus R$ that are integral over $R$.
\end{definition}

The integers are an integrally closed domain, which translates to the statement that monic polynomials with integer coefficients don't have fractional roots.
This notion of integral closure will become important later in the paper in the context of Dedekind domains.

\begin{definition}[Unique factorization domain]
    A commutative ring $R$ is a \textbf{unique factorization domain} if every element factors uniquely into irreducible elements.
\end{definition}

The integers are famously a UFD by the fundamental theorem of arithmetic which states that every integer factors uniquely into a product of primes. But this isn't always the case, for example, in the ring $\Z[i \sqrt 5]$,
\begin{equation}
    \begin{aligned}
        6 & = 2 \cdot 3                     \\
          & = (1 + i \sqrt5) (1 - i \sqrt5)
    \end{aligned}
\end{equation}
where $2$, $3$, $1 + i \sqrt 5$ and $1 - \sqrt 5$ are all irreducible (meaning their only divisors are themselves and units).

\begin{definition}[Principal Ideal Domain]
    A commutative ring $R$ is a \textbf{principal ideal domain} if every ideal is generated by a single element.
\end{definition}

The integers are principal because every ideal is of the form $(a) = \{ 1, \pm 1, \pm2a, \ldots \}$.
The rings $\Z[x]$ and $\Q[x, y]$ are not because of the ideals $(2, x)$ and $(x, y)$ respectively.

\begin{definition}[Noetherian domain]
    \label{def:noetherian}
    A ring $R$ is \textbf{Noetherian} if every ideal is finitely generated.
\end{definition}

We provide (without proof) the following equivalent definitions of Noetherian:

\begin{theorem}
    The following are equivalent
    \begin{enumerate}
        \item $R$ is Noetherian
        \item Every increasing sequence of ideals is eventually constant, i.e., $I_1 \subset I_2 \subset \cdots$ implies that there exists an $M$ such that $I_n = I_m$ for $n, m > M$.
        \item Every non-empty set of ideals $S$ has a ``maximal'' element $M$ such that $M \subset I$ implies $M = I$. There may be multiple such maximal elements.
    \end{enumerate}
\end{theorem}

\begin{figure}
    \centering
    \begin{tikzpicture}[
            node distance = 2em,
            start chain = going right,
            every node/.style = { on chain, join, box, font={\footnotesize}},
            every join/.style = {
                    implies-,
                    double, double equal sign distance,
                    shorten > = 2pt,
                    shorten < = 2pt},
        ]
        \node {Ring};
        \node {Integral \\ Domain};
        \node {Integrally \\ Closed \\ Domain};
        \node {Unique \\ Factorization \\ Domain};
        \node {Principal \\ Ideal \\ Domain};
        \node {Field};
    \end{tikzpicture}
    \caption{The chain of successively stronger ring definitions. Each definition in the chain implies the previous.}
    \label{fig:ring-hierarchy}
\end{figure}

\subsection{Fields}

\begin{definition}[Field]
    A \textbf{field} is a commutative ring $F$ where $0 \neq 1$ and every element $a \in F$ has a multiplicative inverse $a \inv \in F$ such that $a a \inv = e$.
\end{definition}

Fields are often touted as the end of the chain of implications shown in \autoref{fig:ring-hierarchy}. But there's always a bigger fish. For example, the insufficiency of the field $\Q$ becomes very apparent when we consider the polynomial $x^2 + 5$. It wants to factor as
\begin{equation}
    \begin{aligned}
        f(x) & = x^2 + 5                         \\
             & = (x + i \sqrt 5)(x - i \sqrt 5),
    \end{aligned}
\end{equation}
but it can't because $i \sqrt 5 \not \in \Q$. Over $\C$, $f$ does factor completely, but there's a lot of extra stuff in $\C$ that $f$ doesn't care about:
\begin{equation}
    \pi, \quad
    e, \quad
    \sqrt{17}, \quad
    4 + 3 \sqrt[6]{5}, \quad \text{ and } \quad
    e^{2 \pi i / 5}
\end{equation}
to name a few.
In some sense, $f$ would be ``equally happy'' living in a much smaller field without these extraneous elements. This motivates the following definition.

\begin{definition}[Finite field extensions]
    The \emph{field extension} $K(\alpha_1, \ldots, \alpha_n)$ is the smallest field containing both $K$ and each of $\alpha_1, \ldots, \alpha_n$.
\end{definition}

For $f$, the field of concern is
\begin{equation}
    \Q(i \sqrt 5) = \{ a  + b i \sqrt 5 : a, b \in \Q \}.
\end{equation}
This is a number field.

